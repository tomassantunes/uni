\documentclass[10pt]{article}
\usepackage[portuguese]{babel}
\usepackage[utf8]{inputenc}
\usepackage{titlepic}
\usepackage{graphicx}
\title{Inteligência Artificial na Cibersegurança: Deteção de Malware}
\author{Tomás Antunes Nº48511}
\date{Abril 2022}


\begin{document}
\maketitle

\begin{abstract}

Software malicioso, ou malware, é uma das maiores ameaças à cibersegurança. Os ataques informáticos são cada vez mais comuns, todos os dias são criados novos programas maliciosos, com o intuito de causar dano a alguém, a empresas ou até mesmo a governos. Este software malicioso pode danificar a máquina em que está hospedado, pode danificar ficheiros, roubar informações, remover permissões do utilizador ou até executar operações no dispositivo sem a autorização do utilizador. A utilização de várias técnicas de inteligência artificial, como por exemplo Deep Learning e Machine Learning, tem vindo a aumentar, sendo que estas diminuem o trabalho necessário do humano para defender a máquina, fazendo assim uma mais rápida e mais precisa deteção destes softwares maliciosos, podendo assim poupar tempo e dinheiro às entidades atacadas, e evitar danos aos sistemas informáticos. 

O objetivo deste trabalho é explorar a utilização da inteligência artificial na cibersegurança, na área da deteção de malware, e como esta permite que a internet seja um local mais seguro.

\end{abstract}
\newpage

\section{Introdução}

A tecnologia tem vindo a evoluir rapidamente ao longo dos anos, criando a imagem de que vivemos numa sociedade de informação. Estamos constantemente rodeados por tecnologia, que usamos no nosso dia-a-dia. Esta rápida evolução da tecnologia apresenta um grande desafio, que é defender esta nova tecnologia de cibercriminosos. A inteligência artificial pode ajudar neste desafio ao ajudar os operadores de segurança humanos a detetar software malicioso, e assim facilitar o trabalho dos mesmos, poupando tempo e poupando dinheiro. Assim, também a defesa destes sistemas será mais precisa e mais eficaz, pois os operadores sabem exatamente o que “atacar”.


\section{Assinatura de Deteção vs Inteligência Artificial}

Antes da inteligência artificial entrar para a área da segurança, uma das técnicas usadas para detetar software malicioso, era a deteção com base em assinaturas. Este é um método lento, e pouco eficaz, pois é realizado por operadores humanos, que podem cometer erros. O processo de desenvolver uma assinatura de deteção muitas vezes demora semanas, ou seja, o software malicioso podia já ter cumprido a sua missão, e sistemas vitais poderiam estar comprometidos. Mesmo com uma assinatura desenvolvida o atacante apenas precisa de fazer pequenas mudanças no software para este ultrapassar esta. Com a inteligência artificial, não só é mais rápida e mais eficaz a deteção como mesmo que o atacante faça mudanças no código, esta o volte a detetar bastante rapidamente.


\section{Limitações}

Como qualquer outra tecnologia, a inteligência artificial também tem limitações. A deteção de malware é dividida em duas partes, estática e dinâmica, onde ocorre o mapeamento do código do software detetado, sendo a análise dinâmica melhor, comparada com a estática. Uma das limitações do processo dinâmico é por exemplo, algumas operações realizadas pelo utilizador podem parar este processo, fazendo com que este possa falhar algum ou alguns vírus que estejam ativos. Este processo também assume que o estado inicial do ficheiro nunca é alterado o que nem sempre é verdade, podendo assim deixar passar alterações feitas pelo próprio vírus, fazendo com que este tenha maior probabilidade de cumprir a sua missão sem ser detetado.


\section{Conclusão}

Concluindo, a cibersegurança é uma área que está cada vez mais a precisar de ferramentas mais rápidas e mais eficazes, devido à rápida evolução da tecnologia e cada vez haver mais ataques informáticos, e os mesmos sendo cada vez mais sofisticados. A inteligência artificial é uma tecnologia em constante desenvolvimento e melhoria, e uma tecnologia bastante útil na área da cibersegurança. Esta promete ser mais eficaz e mais rápida que o operador humano, e assim poupar recursos e dinheiro aos que usufruírem da mesma, assegurando assim uma internet mais segura.


\nocite{faruk2021malware}
\nocite{libri2020paella}
\nocite{huang2020research}
\nocite{majid2021review}
\nocite{gupta2020ai}
\nocite{abdelsalam2021artificial}
\nocite{alkahtani2022artificial}
\nocite{ferdaos2020smart}
\nocite{poudyal2020ai}
\nocite{vinayakumar2019robust}
\nocite{saad2019curious}
\nocite{scott2017signature}


\bibliographystyle{plain}
\bibliography{referencias}

\end{document}