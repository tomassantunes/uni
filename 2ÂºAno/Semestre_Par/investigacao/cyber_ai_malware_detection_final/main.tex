\documentclass[11pt]{article}
\usepackage[portuguese]{babel}
\usepackage[utf8]{inputenc}
\usepackage{titlepic}
\usepackage{hyperref}
\title{Inteligência Artificial na Cibersegurança: Deteção de Malware}
\author{Tomás Antunes Nº48511}
\date{Junho 2022}


\begin{document}
\maketitle

\begin{abstract}

Software malicioso, ou malware, é uma das maiores ameaças à cibersegurança. Os ataques informáticos são cada vez mais comuns, todos os dias são criados novos programas maliciosos, com o intuito de causar dano a alguém, a empresas ou até mesmo a governos. Este software malicioso pode danificar a máquina em que está hospedado, pode danificar ficheiros, roubar informações, remover permissões do utilizador ou até executar operações no dispositivo sem a autorização do utilizador. A utilização de várias técnicas de inteligência artificial, como por exemplo Deep Learning e Machine Learning, tem vindo a aumentar, sendo que estas diminuem o trabalho necessário do humano para defender a máquina, fazendo assim uma mais rápida e mais precisa deteção destes softwares maliciosos, podendo assim poupar tempo e dinheiro às entidades atacadas, e evitar danos aos sistemas informáticos. 
\paragraph{}
O objetivo deste trabalho é explorar a utilização da inteligência artificial na cibersegurança, na área da deteção de malware, e como esta permite que a internet seja um local mais seguro.

\end{abstract}

\section*{Introdução}

A tecnologia tem vindo a evoluir rapidamente ao longo dos anos, criando a imagem de que vivemos numa sociedade de informação. Estamos constantemente rodeados por tecnologia, que usamos no nosso dia-a-dia. Esta rápida evolução da tecnologia apresenta um grande desafio, que é defender esta nova tecnologia de cibercriminosos. A inteligência artificial pode ajudar neste desafio ao ajudar os operadores de segurança humanos a detetar software malicioso, e assim facilitar o trabalho dos mesmos, poupando tempo e poupando dinheiro. Assim, também a defesa destes sistemas será mais precisa e mais eficaz, pois os operadores sabem exatamente o que “atacar”.\cite{1}

\section*{O que é Inteligência Artificial}
Inteligência Artificial é uma tecnologia que mostra inteligência através de computadores ou outro tipo de máquinas que a use. Esta usa o estudo de algoritmos como forma de aprender e melhorar a sua capacidade de processar o seu ambiente ou de realizar uma determinada tarefa dada. Esta é bastante usada no reconhecimento facial, pois é uma tecnologia bastante eficiente e eficaz no reconhecimento de imagens. Outra das suas muitas aplicações são os motores de busca, onde é utilizada para fornecer os melhores resultados possíveis ao utilizador, normalmente com base nas informações e dados recolhidos sobre o mesmo.
\paragraph{}
No âmbito da cibersegurança a inteligência artificial é utilizada em grande parte para deteção de malware, ou seja, reconhecimento de ameaças.\cite{13}

\section*{O Nascimento da Cibersegurança}
Em 1970 o investigador Bob Thomas criou um programa chamado "Creeper", que conseguia movimentar-se pela rede ARPANET, esta era uma rede de computadores construída em 1969 para a transmissão de dados militares secretos e ligar os departamentos de pesquisa dos Estados Unidos da América, deixando um rasto por onde passava. O inventor do email, Ray Tomlinson, criou um programa chamado "Reaper" que perseguia e apagava o rasto deixado pelo programa de Bob Thomas, tornando-se assim o primeiro anti-vírus a existir, e o nascimento da cibersegurança.\cite{14}\cite{15}

\section*{Deteção de Malware com Inteligência Artificial}
\paragraph{}
\textbf{Deteção de Assinaturas: Humanos vs Inteligência Artificial}

Antes da inteligência artificial entrar para a área da segurança, uma das técnicas usadas para detetar software malicioso, era a deteção com base em assinaturas. Este é um método lento, e pouco eficaz, pois é realizado por operadores humanos, que podem cometer erros. O processo de desenvolver uma assinatura de deteção mitas vezes demora semanas, ou seja, o software malicioso podia já ter cumprido a sua missão, e sistemas vitais poderiam estar comprometidos. Mesmo com uma assinatura desenvolvida, o atacante apenas precisa de fazer pequenas alterações no software para que o malware ultrapasse a mesma.
\paragraph{}
A inteligência artificial, consegue usar a mesma técnica mas de maneira muito mais eficaz. Sendo sendo fornecida uma base de dados, com várias assinaturas/padrões vistos em outros malwares, esta usa algoritmos sofisticados e eficazes para percorrer todos os ficheiros que possam ser maliciosos, à procura destes padrões. Se existir uma correspondência dentro de um ficheiro este é classificado como malicioso e é movido para quarentena, onde um operador humano, ou até mesmo o utilizador da própria máquina, pode decidir o que vai fazer com este ficheiro.\cite{1}\cite{8}\cite{3}

\paragraph{}

\textbf{Deteção de Malware baseada em Anomalias:}
A deteção de malware com base em Anomalias é feita com uma abordagem baseada na classificação para identificar comportamentos normais ou comportamentos estranhos. Esta aborda as limitações da técnica de Deteção de Malware por Assinaturas, na capacidade em que é capaz de avaliar aquilo que é conhecido, tal como, aquilo que não é conhecido.\cite{17}\cite{1}

\paragraph{}

\textbf{Deteção Heurística de Malware:}
A deteção Heurística é a técnica proposto para ultrapassar as desvantagens das técnicas anteriores. Este usa mineração de dados e algoritmos de machine learning para decifrar o comportamento de ficheiros executáveis. A deteção Heurística pode ser feita através de por exemplo, API Calls e OpCodes.
\paragraph{}

\textit{API Calls:}
Quase todos os programas que existem usam API Calls para fazer pedidos ao sistema operativo ou mesmo para pedir algo da internet, como por exemplo a metereologia. Estes podem ser usados para infetar máquinas com malware.

\paragraph{}

\textit{OpCodes:}
OpCode é uma subdivisão da linguagem de máquina que identifica a operação a ser executada. Os programas são sequências de instruções assembly, e cada instrução tem o seu próprio OpCode. Através da análise destes num programa, é possível perceber se tal programa é maligno ou benigno, tendo em conta os padrões de OpCodes presentes no programa.

\cite{18}\cite{19}

\section*{Limitações}
Como qualquer outra tecnologia, a inteligência artificial também tem limitações. A deteção de malware é dividida em duas partes, estática e dinâmica, onde ocorre o mapeamento do código do software detetado, sendo a análise dinâmica melhor, comparada com a estática. A deteção de malware estática pode falhar, tendo em conta as ferramentas que existem hoje em dia, como por exemplo, ferramentas com a capacidade de alterar ou reescrever binário, permitindo assim mascarar programas para que estes passem por benignos, ou até apenas porque falha a detetar uma anomalia que nunca foi detetada anteriormente. Uma das limitações do processo dinâmico é por exemplo, algumas operações realizadas pelo utilizador podem parar este processo, fazendo com que este possa falhar algum ou alguns vírus que estejam ativos. Este processo também assume que o estado inicial do ficheiro nunca é alterado o que nem sempre é verdade, podendo assim deixar passar alterações feitas pelo próprio vírus, fazendo com que este tenha maior probabilidade de cumprir a sua missão sem ser detetado.\cite{1}\cite{20}


\section*{Conclusão}

Concluindo, a cibersegurança é uma área que está cada vez mais a precisar de ferramentas mais rápidas e mais eficazes, devido à rápida evolução da tecnologia e cada vez haver mais ataques informáticos, e os mesmos sendo cada vez mais sofisticados. A inteligência artificial é uma tecnologia em constante desenvolvimento e melhoria, e uma tecnologia bastante útil na área da cibersegurança. Esta promete ser mais eficaz e mais rápida que o operador humano, e assim poupar recursos e dinheiro aos que usufruírem da mesma, assegurando assim uma internet mais segura.

\newpage

\bibliographystyle{plain}
\bibliography{referencias}

\end{document}
