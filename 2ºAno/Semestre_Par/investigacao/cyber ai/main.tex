\documentclass{article}
\usepackage[portuguese]{babel}
\usepackage[utf8]{inputenc}
\title{Inteligência Artificial na Cibersegurança}
\author{Tomás Antunes Nº48511}
\date{Março 2022}
\begin{document}
\maketitle
\begin{abstract}

O tema que proponho para o trabalho de investigação é o uso da inteligência artificial na cibersegurança. A inteligência artificial é uma tecnologia que tem vindo a evoluir bastante ao longo do tempo, sendo esta bastante útil em várias áreas da informação e tecnologia. Nos últimos anos, começou a ser utilizada na área da segurança, de modo a ajudar os humanos a combater o crime online. Esta tecnologia aprende com experiências passadas, reconhecendo padrões nos dados recolhidos. Um dos grandes benefícios, e uma das razões que o setor privado tem adotado tanto a inteligência artificial como auxílio na cibersegurança, é devido ao facto da sua capacidade de processar informação bastante mais rápido que os humanos, e assim poupar tempo e dinheiro às empresas. Esta pode detetar zero-day malware, priorizar ameaças e monitorizar comportamentos anômalos, por exemplo a forma como uma password é escrita, ou onde é que o utilizador está a fazer log in, estas pequenas diferenças de comportamento são detetadas pela tecnologia, possibilitando assim ao segurança humano uma defesa mais precisa e eficaz. Com o recente aumento de ataques cibernéticos, a inteligência artificial pode fazer com que a defesa seja mais eficaz, mais inteligente e mais rápida, com por exemplo, sistemas de cibersegurança automáticos, que recolhem dados de ataques, e que usando esses dados conseguem defender e negar futuros ataques sem qualquer intervenção humana. 

Concluindo, a inteligência artificial é uma tecnologia bastante interessante, e que está a evoluir rapidamente, tornando-se mais útil e mais confiável. Esta ajuda as equipas de cibersegurança e cria uma relação humano-máquina que beneficia muito o seu sucesso, e tornando a internet um local mais seguro.



\end{abstract}

\cite{kuzlu2021role}
\cite{laato2020ai}
\cite{sarker2021ai}
\cite{lazic2019benefit}
\cite{stevens2020knowledge}





\bibliographystyle{plain}
\bibliography{references.bib}

\end{document}